\documentclass{template/han}

\title{
Casus RDI 2019 \newline
\large Video-on-demand internetplatform ODISEE
}
\def\docenten{
Marco Engelbart \newline
Mark Giessen \newline
Bram Laumans
}
\def\vak{
ADB-DT
}
\author{
Nick Hartjes (423064) \newline
Marc Groenhout (527698) \newline
Joey Stoffels (609589)
}
\date{\today}

\begin{document}
    \newpage
    \maketitle

    %=================o ~~
    \newpage
    \tableofcontents

    %=================o ~~
    \newpage
    \section{Opdracht 1: Bevragingen}
    \label{opdracht1}
    \paragraph{
        Maak voor iedere query die je schrijft minimaal ��n alternatieve implementatie indien mogelijk en vergelijk de queryplannen.
        Kies op basis van het queryplan, performance en onderhoudbaarheid van de code de juiste query en verklaar jouw keuze.
    }
    \subsection{Opdracht 01}

Geef van een film, de hele reeks waar die bij hoort met volgnummer en in de juiste volgorde.
Indien hij niet in een reeks zit, is de lijst gewoon ��n lang met volgnummer 1.
Dit moet ��n statement worden die van een variabel ID de reeks geeft zoals onderstaand voorbeeld:
DECLARE @MovieInReeks INT = 207989;


\begin{lstlisting}
-- jouw statement hier levert onderstaand resultaat:

ITEM_ID		TITLE				Volgnummer
207992		Matrix, The				1
207989		Matrix Reloaded, The			2
207991		Matrix Revolutions, The		3
\end{lstlisting}

\subsubsection{Versie 01}

TODO: Marc

\lstinputlisting[language=SQL]{sql/marc/opdracht-01.sql}
\begin{figure}
    \centering
    \includegraphics[width=1\textwidth]{image/marc/opdracht-01.PNG}
    \caption{Queryplan Opdracht 01 Versie 01}
\end{figure}

\subsubsection{Versie 02}
TODO: Joey

\lstinputlisting[language=SQL]{sql/joey/opdracht-01.sql}
\begin{figure}
    \centering
    \includegraphics[width=1\textwidth]{image/joey/opdracht-01.PNG}
    \caption{Queryplan Opdracht 01 Versie 02}
\end{figure}


\subsubsection{Versie 03}

    De intentie van deze query was om geen CTE te gebruiken. In dit alternatief maken we gebruik van een tijdelijke tabel.
    Na het aanmaken van van de tijdelijke tabel worden er 2 while loops gedraaid. Bij de eerste loop word er gekeken of er een vorige film is.
    Als die er is, word die opgeslagen in de tijdelijke tabel, en herhaald hij de loop tot het punt komt dat er geen vorige film meer is.
    Hetzelfde doet hij met vervolg films, als deze er zijn word die ook in de tijdelijke tabel opgeslagen.
    Uiteindelijk word de tijdelijke tabel uitgevraagd, en krijg je de lijst met films te zien.
    Als laatste word de tijdelijke tabel verwijderd.

\lstinputlisting[language=SQL]{sql/nick/opdracht-01.sql}

\subsubsection{Conclusie}

\begin{tabular}{ || l | l | l | l | l | l | l | l | l | l | l | l | 1 | 1 | l | 1 | 1 || }
    \hline
    \textbf{Statement} & \textbf{Est Cost \%} & \textbf{Compile Time} & \textbf{Duration} &
    \textbf{CPU} & \textbf{Est CPU Cost \%} &
    \textbf{Est IO Cost \%} \\
    \hline
    \hline
    Versie01  & 0,0\%  & 12  & 40  & 40  & 4,9\% & 0,0\% \\
    \hline
    Versie02  & 100,0\%  & 20  & 76  & 75  & 95,1\% & 100,0\%  \\
    \hline
\end{tabular}
\newline
\newline
\begin{tabular}{ || l | l | l | l | l | l | l | l | l | l | l | l | 1 | 1 | l | 1 | 1 || }
    \hline
    \textbf{Statement} &  \textbf{Est Rows} & \textbf{Actual Rows} & \textbf{RID Lookups} &
    \textbf{Parallel} & \textbf{Sort} &
    \textbf{Table Scan} & \textbf{Hash Match} \\
    \hline
    \hline
    Versie01  & 77.765  & 3  & 2  & 0  & 1  & 1  & 2 \\
    \hline
    Versie02  & 1.157.780  & 3  & 3  & 5  & 1  & 1  & 0 \\
    \hline
\end{tabular}


HIer komt dus een lang verhaal over welke query het beste is, en waarom

    \clearpage
    \subsection{Opdracht 02}

\paragraph{
Geef het statement dat per land het maandgebruik over 12 maanden geeft met daarbij het percentage dat die maand uitmaakt van het totaal van die 12 maanden.
Hier zijn twee interpretaties mogelijk: per land de afgelopen 12 maanden in rijen, hier onder de uitdraai voor b.v. �Netherlands� uitgevoerd in april 2019.
}

\begin{lstlisting}
    Year	Month	ItemsPerMonth	PercentageOfTotal
    2018	4	60		4.44%
    2018	5	60		4.44%
    2018	6	60		4.44%
    2018	7	56		4.14%
    2018	8	132		9.76%
    2018	9	155		11.46%
    2018	10	141		10.43%
    2018	11	138		10.21%
    2018	12	148		10.95%
    2019	1	137		10.13%
    2019	2	117		8.65%
    2019	3	148		10.95%
\end{lstlisting}

\lstinputlisting[language=SQL]{sql/nick/opdracht-02a.sql}
\lstinputlisting[language=SQL]{sql/marc/opdracht-02a.sql}
\lstinputlisting[language=SQL]{sql/joey/opdracht-02a.sql}
    \clearpage
    \subsection{Opdracht 02b}
Een mooier alternatief is een statement dat per land de percentages en totalen geeft over een jaar in kolommen.

\begin{lstlisting}
    -- jouw statement hier levert b.v. onderstaand resultaat voor 2017

    Countryname  January	Feburary  March    April    May    June    July    August  September  October  November  December  TotalItems
    Chile         9.76%    8.22%   10.06%    9.07%   9.26%   8.38%   8.91%    7.75%    7.28%     7.09%     7.53%    6.68%     3638
    Greece        9.67%    7.69%    9.06%    8.49%   8.49%   7.92%   9.06%    8.49%    8.40%     7.78%     7.78%    7.17%     2120
    Poland        9.93%    7.94%    9.41%    8.82%   8.82%   8.24%   8.24%    7.72%    7.72%     7.72%     7.72%    7.72%     1360
    Netherlands   8.38%    7.26%    8.94%    8.38%   8.38%   7.82%   8.94%    8.38%    8.38%     8.38%     8.38%    8.38%      716
\end{lstlisting}

\subsubsection{Versie 01}
TODO: Nick
\lstinputlisting[language=SQL]{sql/nick/opdracht-01-02b-v1.sql}
\begin{figure}[H]
    \centering
    \includegraphics[width=1\textwidth]{image/nick/opdracht-02b.PNG}
    \caption{Queryplan Opdracht 02b Versie 01}
\end{figure}

\subsubsection{Versie 02}
In onderstaande versie wordt er gebruik gemaakt van een PIVOT. Bij een PIVOT worden rijen omgezet naar kolommen. Indien er geen verkopen
zijn in een bepaalde maand, dan wordt er een standaard waarde van 0.00\% getoond. Door in de FROM clausule de juiste set met gegevens
aan te leveren, wordt in het SELECT statement enkel en alleen nog de gewenste informatie uit de juiste kolommen opgevraagd.
\lstinputlisting[language=SQL]{sql/marc/opdracht-01-02b.sql}
\begin{figure}[H]
    \centering
    \includegraphics[width=1\textwidth]{image/marc/opdracht-02b.PNG}
    \caption{Queryplan Opdracht 02b Versie 02}
\end{figure}

\subsection{Conclusie}
Versie 01 presteert met 22\% van de totale duur van de gehele batch aanzienlijk sneller dan versie 02 welke 78\% van de duur nodig heeft.
Uit het queryplan wordt duidelijk dat versie 02 in het beginstadium net iets beter presteert. Deze voert een 'Index Seek (NonClustered)' uit
voor 2 van de 11 records in de User tabel. Hetzelfde geldt voor de 'RID Lookup (Heap)' welke eveneens voor 2 van de 11 records informatie
uit de User tabel ophaalt. Na deze stap doorloopt versie 01 echter een eenvoudig proces zonder extra 'Nested Loops' en de extra ballast
welke hieraan vooraf gaat zoals een 'Filter', 'Clustered Index Scan' en een 'Table Scan'. Derhalve kan worden gesteld dat versie 01
beter preseteert dan versie 02. De onderhoudbaarheid van versie 01 is eveneens beter dan die van versie 02, niet alleen omdat de structuur
helder en overzichtelijk is, maar ook vanwege het gebruik van een CTE waardoor een duidelijke resultset terug wordt gegeven.\\
\\
\begin{tabular}{ || l | l | l | l | l | l | l | l | l | l | l | l | 1 | 1 | l | 1 | 1 || }
    \hline
    \textbf{Statement} & \textbf{Est Cost \%} & \textbf{Compile Time} & \textbf{Duration} &
    \textbf{CPU} & \textbf{Est CPU Cost \%} &
    \textbf{Est IO Cost \%} \\
    \hline
    \hline
    Versie01  & 22,5\%  & 12  & 2  & 2  & 9,8\% & 28,9\% \\
    \hline
    Versie02  & 77,5\%  & 35  & 8  & 3  & 90,2\% & 71,1\%  \\
    \hline
\end{tabular}
\newline
\newline
\begin{tabular}{ || l | l | l | l | l | l | l | l | l | l | l | l | 1 | 1 | l | 1 | 1 || }
    \hline
    \textbf{Statement} &  \textbf{Est Rows} & \textbf{Actual Rows} & \textbf{RID Lookups} &
    \textbf{Parallel} & \textbf{Sort} &
    \textbf{Table Scan} & \textbf{Hash Match} \\
    \hline
    \hline
    Versie01  & 2  & 2  & 1  & 0  & 1  & 0  & 0 \\
    \hline
    Versie02  & 2  & 2  & 3  & 0  & 1  & 1  & 0 \\
    \hline
\end{tabular}
    \clearpage

    \section{Opdracht 2: Constraints}
    \paragraph{
    Implementeer onderstaande constraints.
    Het kan zijn dat je een aantal al tijdens de casus DDDQ heb ge�dentificeerd en/of zelfs al gemaakt.
    Lever in dat geval die code weer in voor deze casus. Indien je de constraints procedureel oplost, maak je zowel een Stored Procedure als een After Trigger.
    Zorg voor nette error handling in jouw code en schrijf een complete testset voor beide constraint implementaties.
    Leg uit of de SP of Trigger jouw voorkeur heeft en uiteraard waarom.
    Als je bij een constraint vindt dat een Instead Of Trigger de betere variant is, maak je die ook en leg je uit waarom dit de beste oplossing is voor deze constraint.

    }
    \label{opdracht2}
    \subsection{Opdracht 01}

Geef van een film, de hele reeks waar die bij hoort met volgnummer en in de juiste volgorde.
Indien hij niet in een reeks zit, is de lijst gewoon ��n lang met volgnummer 1.
Dit moet ��n statement worden die van een variabel ID de reeks geeft zoals onderstaand voorbeeld:
DECLARE @MovieInReeks INT = 207989;


\begin{lstlisting}
-- jouw statement hier levert onderstaand resultaat:

ITEM_ID		TITLE				Volgnummer
207992		Matrix, The				1
207989		Matrix Reloaded, The			2
207991		Matrix Revolutions, The		3
\end{lstlisting}

\subsubsection{Versie 01}

TODO: Marc

\lstinputlisting[language=SQL]{sql/marc/opdracht-01.sql}
\begin{figure}
    \centering
    \includegraphics[width=1\textwidth]{image/marc/opdracht-01.PNG}
    \caption{Queryplan Opdracht 01 Versie 01}
\end{figure}

\subsubsection{Versie 02}
TODO: Joey

\lstinputlisting[language=SQL]{sql/joey/opdracht-01.sql}
\begin{figure}
    \centering
    \includegraphics[width=1\textwidth]{image/joey/opdracht-01.PNG}
    \caption{Queryplan Opdracht 01 Versie 02}
\end{figure}


\subsubsection{Versie 03}

    De intentie van deze query was om geen CTE te gebruiken. In dit alternatief maken we gebruik van een tijdelijke tabel.
    Na het aanmaken van van de tijdelijke tabel worden er 2 while loops gedraaid. Bij de eerste loop word er gekeken of er een vorige film is.
    Als die er is, word die opgeslagen in de tijdelijke tabel, en herhaald hij de loop tot het punt komt dat er geen vorige film meer is.
    Hetzelfde doet hij met vervolg films, als deze er zijn word die ook in de tijdelijke tabel opgeslagen.
    Uiteindelijk word de tijdelijke tabel uitgevraagd, en krijg je de lijst met films te zien.
    Als laatste word de tijdelijke tabel verwijderd.

\lstinputlisting[language=SQL]{sql/nick/opdracht-01.sql}

\subsubsection{Conclusie}

\begin{tabular}{ || l | l | l | l | l | l | l | l | l | l | l | l | 1 | 1 | l | 1 | 1 || }
    \hline
    \textbf{Statement} & \textbf{Est Cost \%} & \textbf{Compile Time} & \textbf{Duration} &
    \textbf{CPU} & \textbf{Est CPU Cost \%} &
    \textbf{Est IO Cost \%} \\
    \hline
    \hline
    Versie01  & 0,0\%  & 12  & 40  & 40  & 4,9\% & 0,0\% \\
    \hline
    Versie02  & 100,0\%  & 20  & 76  & 75  & 95,1\% & 100,0\%  \\
    \hline
\end{tabular}
\newline
\newline
\begin{tabular}{ || l | l | l | l | l | l | l | l | l | l | l | l | 1 | 1 | l | 1 | 1 || }
    \hline
    \textbf{Statement} &  \textbf{Est Rows} & \textbf{Actual Rows} & \textbf{RID Lookups} &
    \textbf{Parallel} & \textbf{Sort} &
    \textbf{Table Scan} & \textbf{Hash Match} \\
    \hline
    \hline
    Versie01  & 77.765  & 3  & 2  & 0  & 1  & 1  & 2 \\
    \hline
    Versie02  & 1.157.780  & 3  & 3  & 5  & 1  & 1  & 0 \\
    \hline
\end{tabular}


HIer komt dus een lang verhaal over welke query het beste is, en waarom

    \clearpage

    \section{Opdracht 3: Transactie management en Concurrency}
    \paragraph{
        In zowel jouw SP's als Triggers maak je correct gebruik van Transacties.
        Kies het juiste isolation level bij een implementatie en leg uit waarom dat het juiste level is.
        Leg hierbij uit welke problemen zich kunnen voordoen in het default isolation level, hoe locking in die situatie werkt en hoe die d.m.v. het isolation level aangepast wordt.
    }
    \label{opdracht3}
    \subsection{Opdracht 01}

Geef van een film, de hele reeks waar die bij hoort met volgnummer en in de juiste volgorde.
Indien hij niet in een reeks zit, is de lijst gewoon ��n lang met volgnummer 1.
Dit moet ��n statement worden die van een variabel ID de reeks geeft zoals onderstaand voorbeeld:
DECLARE @MovieInReeks INT = 207989;


\begin{lstlisting}
-- jouw statement hier levert onderstaand resultaat:

ITEM_ID		TITLE				Volgnummer
207992		Matrix, The				1
207989		Matrix Reloaded, The			2
207991		Matrix Revolutions, The		3
\end{lstlisting}

\subsubsection{Versie 01}

TODO: Marc

\lstinputlisting[language=SQL]{sql/marc/opdracht-01.sql}
\begin{figure}
    \centering
    \includegraphics[width=1\textwidth]{image/marc/opdracht-01.PNG}
    \caption{Queryplan Opdracht 01 Versie 01}
\end{figure}

\subsubsection{Versie 02}
TODO: Joey

\lstinputlisting[language=SQL]{sql/joey/opdracht-01.sql}
\begin{figure}
    \centering
    \includegraphics[width=1\textwidth]{image/joey/opdracht-01.PNG}
    \caption{Queryplan Opdracht 01 Versie 02}
\end{figure}


\subsubsection{Versie 03}

    De intentie van deze query was om geen CTE te gebruiken. In dit alternatief maken we gebruik van een tijdelijke tabel.
    Na het aanmaken van van de tijdelijke tabel worden er 2 while loops gedraaid. Bij de eerste loop word er gekeken of er een vorige film is.
    Als die er is, word die opgeslagen in de tijdelijke tabel, en herhaald hij de loop tot het punt komt dat er geen vorige film meer is.
    Hetzelfde doet hij met vervolg films, als deze er zijn word die ook in de tijdelijke tabel opgeslagen.
    Uiteindelijk word de tijdelijke tabel uitgevraagd, en krijg je de lijst met films te zien.
    Als laatste word de tijdelijke tabel verwijderd.

\lstinputlisting[language=SQL]{sql/nick/opdracht-01.sql}

\subsubsection{Conclusie}

\begin{tabular}{ || l | l | l | l | l | l | l | l | l | l | l | l | 1 | 1 | l | 1 | 1 || }
    \hline
    \textbf{Statement} & \textbf{Est Cost \%} & \textbf{Compile Time} & \textbf{Duration} &
    \textbf{CPU} & \textbf{Est CPU Cost \%} &
    \textbf{Est IO Cost \%} \\
    \hline
    \hline
    Versie01  & 0,0\%  & 12  & 40  & 40  & 4,9\% & 0,0\% \\
    \hline
    Versie02  & 100,0\%  & 20  & 76  & 75  & 95,1\% & 100,0\%  \\
    \hline
\end{tabular}
\newline
\newline
\begin{tabular}{ || l | l | l | l | l | l | l | l | l | l | l | l | 1 | 1 | l | 1 | 1 || }
    \hline
    \textbf{Statement} &  \textbf{Est Rows} & \textbf{Actual Rows} & \textbf{RID Lookups} &
    \textbf{Parallel} & \textbf{Sort} &
    \textbf{Table Scan} & \textbf{Hash Match} \\
    \hline
    \hline
    Versie01  & 77.765  & 3  & 2  & 0  & 1  & 1  & 2 \\
    \hline
    Versie02  & 1.157.780  & 3  & 3  & 5  & 1  & 1  & 0 \\
    \hline
\end{tabular}


HIer komt dus een lang verhaal over welke query het beste is, en waarom

    \clearpage

    \section{Opdracht 4: Indexeren}
    \paragraph{
        Geef aan waar jij denkt dat indexen nuttig kunnen zijn.
        Toon het verschil aan tussen de situatie met en zonder index en verklaar de verschillen in performance en queryplannen.
        Mogelijk veranderd je keuze van query als je gebruik maakt van indexen, geef dat dan aan.
    }
    \label{opdracht4}
    \subsection{Opdracht 01}

Geef van een film, de hele reeks waar die bij hoort met volgnummer en in de juiste volgorde.
Indien hij niet in een reeks zit, is de lijst gewoon ��n lang met volgnummer 1.
Dit moet ��n statement worden die van een variabel ID de reeks geeft zoals onderstaand voorbeeld:
DECLARE @MovieInReeks INT = 207989;


\begin{lstlisting}
-- jouw statement hier levert onderstaand resultaat:

ITEM_ID		TITLE				Volgnummer
207992		Matrix, The				1
207989		Matrix Reloaded, The			2
207991		Matrix Revolutions, The		3
\end{lstlisting}

\subsubsection{Versie 01}

TODO: Marc

\lstinputlisting[language=SQL]{sql/marc/opdracht-01.sql}
\begin{figure}
    \centering
    \includegraphics[width=1\textwidth]{image/marc/opdracht-01.PNG}
    \caption{Queryplan Opdracht 01 Versie 01}
\end{figure}

\subsubsection{Versie 02}
TODO: Joey

\lstinputlisting[language=SQL]{sql/joey/opdracht-01.sql}
\begin{figure}
    \centering
    \includegraphics[width=1\textwidth]{image/joey/opdracht-01.PNG}
    \caption{Queryplan Opdracht 01 Versie 02}
\end{figure}


\subsubsection{Versie 03}

    De intentie van deze query was om geen CTE te gebruiken. In dit alternatief maken we gebruik van een tijdelijke tabel.
    Na het aanmaken van van de tijdelijke tabel worden er 2 while loops gedraaid. Bij de eerste loop word er gekeken of er een vorige film is.
    Als die er is, word die opgeslagen in de tijdelijke tabel, en herhaald hij de loop tot het punt komt dat er geen vorige film meer is.
    Hetzelfde doet hij met vervolg films, als deze er zijn word die ook in de tijdelijke tabel opgeslagen.
    Uiteindelijk word de tijdelijke tabel uitgevraagd, en krijg je de lijst met films te zien.
    Als laatste word de tijdelijke tabel verwijderd.

\lstinputlisting[language=SQL]{sql/nick/opdracht-01.sql}

\subsubsection{Conclusie}

\begin{tabular}{ || l | l | l | l | l | l | l | l | l | l | l | l | 1 | 1 | l | 1 | 1 || }
    \hline
    \textbf{Statement} & \textbf{Est Cost \%} & \textbf{Compile Time} & \textbf{Duration} &
    \textbf{CPU} & \textbf{Est CPU Cost \%} &
    \textbf{Est IO Cost \%} \\
    \hline
    \hline
    Versie01  & 0,0\%  & 12  & 40  & 40  & 4,9\% & 0,0\% \\
    \hline
    Versie02  & 100,0\%  & 20  & 76  & 75  & 95,1\% & 100,0\%  \\
    \hline
\end{tabular}
\newline
\newline
\begin{tabular}{ || l | l | l | l | l | l | l | l | l | l | l | l | 1 | 1 | l | 1 | 1 || }
    \hline
    \textbf{Statement} &  \textbf{Est Rows} & \textbf{Actual Rows} & \textbf{RID Lookups} &
    \textbf{Parallel} & \textbf{Sort} &
    \textbf{Table Scan} & \textbf{Hash Match} \\
    \hline
    \hline
    Versie01  & 77.765  & 3  & 2  & 0  & 1  & 1  & 2 \\
    \hline
    Versie02  & 1.157.780  & 3  & 3  & 5  & 1  & 1  & 0 \\
    \hline
\end{tabular}


HIer komt dus een lang verhaal over welke query het beste is, en waarom

    \clearpage

    %=================o ~~
    \newpage
    \section{Bronvermelding}
    \label{bronvermelding}

    \nocite{*}

\end{document}